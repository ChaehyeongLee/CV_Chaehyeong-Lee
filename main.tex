\documentclass[11pt,usletter,sans]{moderncv}

% Style and color
\moderncvstyle{banking}   % casual, classic, oldstyle, banking
\moderncvcolor{blue}      % blue, orange, green, red, purple, grey, black
\setlength{\hintscolumnwidth}{4cm} 
\renewcommand{\arraystretch}{1.1}  
\usepackage{academicons}
\definecolor{orcidlogocol}{HTML}{A6CE39}

% Adjust margins
\usepackage[scale=0.85]{geometry}
\newcommand{\doi}[1]{\href{https://doi.org/#1}{doi:#1}}

% Personal data
\name{Chaehyeong}{Lee}
\address{Boulder, Colorado, USA}{}
\phone[mobile]{+1~(303)~258-6841}
\email{Chaehyeong.Lee@colorado.edu}
\homepage{sites.google.com/yonsei.ac.kr/hyeong}   
\extrainfo{\href{https://orcid.org/0009-0005-3110-9839}{\textcolor{orcidlogocol}{\aiOrcid}~https://orcid.org/0009-0005-3110-9839}}

\begin{document}
\makecvtitle

%------------------------------------------------------------------------------
\section{Research Interests}
\cvitem{}{Ocean dynamics and climate sciences: ocean heat budget; ocean's role in climate systems.}
\cvitem{}{Geofluid dynamics: upper-ocean mixing processes in the frequency domain.}

%------------------------------------------------------------------------------
\section{Education}
\cventry{Aug.~2024 -- present}
        {Ph.D. in Atmospheric and Oceanic Sciences}
        {University of Colorado Boulder}
        {Boulder, CO, USA}
        {}
        {Advisors: Dr. Donata Giglio \& Dr. Aneesh Subramanian}

\vspace{0.5em}

\cventry{Mar.~2022 -- Aug.~2023}
        {M.S. in Atmospheric Science}
        {Yonsei University}
        {Seoul, Rep. of Korea}
        {}
        {Advisor: Dr. Hajoon Song}

\vspace{0.5em}

\cventry{Mar.~2016 -- Feb.~2022}
        {B.S. in Atmospheric Science}
        {Yonsei University}
        {Seoul, Rep. of Korea}
        {}
        {}

%------------------------------------------------------------------------------
\section{Publications}

\subsection{In Progress}
\cvitem{}{
\textbf{Lee, C.}, Giglio, D., Subramanian, A. C., Han, W., Capotondi, A., Du, D., \& Molod, A.
\textit{Assessing the impact of sea surface salinity assimilation on extreme event prediction in NASA GEOS-S2S v2 model}.
}
\cvitem{}{
\textbf{Lee, C.}, Giglio, D., \& Subramanian, A. C. 
\textit{Bridging gaps in the upper-ocean heat budget between observations and climate models: a frequency-domain perspective}.
}

\vspace{1em} % Adds space between subsections

\subsection{Published}
\cvitem{}{
\textbf{Lee, C.}, Song, H., Choi, Y., Cho, A., \& Marshall, J. (2025). 
Observed multi-decadal increase in the surface ocean’s thermal inertia. 
\textit{Nature Climate Change}, 1–7. 
\doi{10.1038/s41558-025-02245-w}
}

%------------------------------------------------------------------------------
\section{Research Experience}
\cventry{Aug.~2024 -- present}
        {Research Assistant}
        {Giglio's Research Group, CU Boulder}
        {Boulder, CO, USA}
        {}
        {
        Developing methods to improve NASA GEOS-S2S v2 simulations via sea-surface salinity assimilation; Filling gaps in upper-ocean mixing processes between observations and models through frequency-domain analysis.
        }

\vspace{0.5em}

\cventry{Aug.~2024 -- present}
        {Research Assistant}
        {Climate Processes and Predictability Group, CU Boulder}
        {Boulder, CO, USA}
        {}
        {
        % Developing methods to improve NASA GEOS-S2S v2 simulations via sea-surface salinity assimilation; Filling gaps in upper-ocean mixing processes between observations and models through frequency-domain analysis.
        }

\vspace{0.5em}

\cventry{Dec.~2020 -- Aug.~2024}
        {Research Assistant}
        {Air-Sea Modeling Lab, Yonsei University}
        {Seoul, Rep. of Korea}
        {}
        {
        Analyzed changes in the upper-ocean thermal state using SST observations; examined hysteresis of thermal memory under CESM 4$\times$CO\textsubscript{2} experiments.
        }

%------------------------------------------------------------------------------

\section{Teaching Experience}
\cventry{Spring 2026}
        {Teaching Assistant}
        {Data Science Laboratory, taught by Dr. Donata Giglio at CU Boulder}
        {Boulder, CO, USA}
        {}
        {}
        
\vspace{0.5em}

\cventry{Spring 2023}
        {Teaching Assistant}
        {Climate \& Civilization, taught by Dr. Yign Noh at Yonsei Univ.}
        {Seoul, Rep. of Korea}
        {}
        {}
        
\vspace{0.5em}

\cventry{Fall 2022}
        {Teaching Assistant}
        {Physical Oceanography, taught by Dr. Hajoon Song at Yonsei Univ.}
        {Seoul, Rep. of Korea}
        {}
        {}

%------------------------------------------------------------------------------

\section{Awards \& Scholarships}
\cvitem{2025}
       {ATOC Fellowship, Department of Atmospheric and Oceanic Sciences, University of Colorado Boulder}
\cvitem{2025}
       {Academic Research Grants (GCP research credits), Google LLC}
\cvitem{2024}
       {Outstanding Thesis Award, Yonsei University Graduate School, Yonsei University}
\cvitem{2022--2023}
       {Full tuition merit scholarship (for the top 2 graduate students), Yonsei University}
\cvitem{2022}
       {High Honors for academic performance, Yonsei University}
\cvitem{2020--2021}
       {Jilli Scholarship for academic performance, Yonsei University}

%------------------------------------------------------------------------------
\section{Patent}
\cvitem{}
       {
       Song, H., \& \textbf{Lee, C.} (2025). \textit{Evaluation system and method of persistence of SST anomalies using autocorrelation coefficient and arctangent regressive model}. 
       Rep. of Korea Patent \#1028135790000.
       \doi{10.8080/1020220157159}
       }

%------------------------------------------------------------------------------
\section{Invited Talk}
    
\cventry{Dec.~2025}                 % 1: date (right)
        {\parbox[t]{0.8\textwidth}{\textit{Assessing the Impact of Satellite Sea Surface Salinity Assimilation on Vertical Structure of the Upper Ocean in the NASA GEOS-S2S 2.}}}
        {NASA Salinity Telecon}      % 3: event name (bold, left)
        {Virtual Meeting}             % 4: location (right)
        {}                           % 5: empty (grade)
        {}                           % 6: description [cite: 49, 50, 63]        

%------------------------------------------------------------------------------
\section{Conferences}

\cventry{Feb.~2026}                 % 1: date (right)
        {\parbox[t]{0.8\textwidth}{\textit{Assessing the Impact of Satellite Sea Surface Salinity Assimilation on the Upper Ocean Thermal State in the NASA GEOS S2S-v2 Model.}}}
        {OSM 2026 (poster)}   % 3: event name (bold, left)
        {Glasgow, Scotland}           % 4: location (right)
        {}
        {}
        {\emph{\textbf{Lee, C.}, Giglio, D., \& Subramanian, A. C.}}
        
\vspace{0.5em}

\cventry{Dec.~2022}                 % 1: date (right)
        {\textit{The increasing trend of persistence of sea surface temperature in the past 40 years.}}
        {AGU Fall Meeting (poster)}   % 3: event name (bold, left)
        {Chicago, IL, USA}            % 4: location (right)
        {}
        {\emph{\textbf{Lee, C.}, Song, H., Cho, A., \& Tak, Y.}}
        
\vspace{0.5em}

\cventry{Jun.~2022}                 % 1: date (right)
        {\textit{Increasing persistence of SST anomalies and duration of marine heatwaves.}}
        {Korean Society of Oceanography Spring Conference (talk)} % 3: event name (bold, left)
        {Jeju, Rep. of Korea}        % 4: location (right)
        {}
        {\emph{\textbf{Lee, C.}, Song, H., Cho, A., \& Tak, Y.}}

%------------------------------------------------------------------------------
\section{Workshops}

\cventry{Jan.~2022}                 % 1: date (right)
        {Organized by the Korea Meteorological Administration}
        {User Training for the Glosea 6 Climate Prediction Model}
        {Jeju, Rep. of Korea}        % 4: location (right)
        {}                           % 5: empty (grade)
        {}                           % 6: description [cite: 70, 74, 75]

\vspace{0.5em}

\cventry{Jan.~2022}                 % 1: date (right)
        {NVIDIA Deep Learning Institute}
        {Deep Learning Training: Fundamentals of Deep Learning}
        {Gonju, Rep. of Korea}       % 4: location (right)
        {}                           % 5: empty (grade)
        {}                           % 6: description [cite: 71, 72, 76, 77]

%------------------------------------------------------------------------------
\section{Service}
\cvitem{Peer Reviewer}{\textit{Journal of Climate}}

%------------------------------------------------------------------------------
\section{Technical Skills}
\cvitem{Programming}{Python (xarray, dask, Pangeo), Julia (Oceananigans)}
\cvitem{HPC}{Parallel/distributed computing, NCAR Casper/Derecho clusters}
\cvitem{Tools}{Git, Linux shell scripting, LaTeX}

%------------------------------------------------------------------------------
\end{document}